\documentclass[12pt]{article}
\usepackage[pdftex]{graphicx}
\usepackage{amsmath}
\usepackage{amssymb}
\pagestyle{empty}

\topmargin -0.6in
\headsep 0.40in
\oddsidemargin 0.0in
\textheight 9.0in
\textwidth 6.5in

\newcommand{\econst}{\mathrm{e}}
\newcommand{\diff}{\mathrm{d}}
\newcommand{\dwrt}[1]{\frac{\diff}{\diff #1}}

\everymath={\displaystyle}


\begin{document}

\noindent
Math 425 \qquad 
Applied and Computational Linear Algebra \qquad
Spring 2022
\vskip 5pt
\noindent
hw3 , \underline{PART A}: {\bf due on iLearn by 12:30pm on Thursday, March 3}
\vskip 10pt
\noindent
1. Let $T({\bf x}) = A{\bf x}$. If  
$A=\begin{bmatrix}
1 & -3 & ~~2\\
0 & ~~1 & -4\\
3 & -5 & -9
\end{bmatrix}$, and ${\bf b} = \begin{bmatrix}~~6\\ -7 \\ -9 \end{bmatrix}$, find a vector ${\bf x}$
whose image under $T$ is ${\bf b}$, and determine whether ${\bf x}$ is unique.
\vskip 10pt
\noindent
2. Let
$A=\begin{bmatrix}
~~1 & 3 & 9 & ~~2\\
~~1 & 0 & 3 & -4\\
~~0 & 1 & 2 & ~~3\\
-2 & 3 & 0 & ~~5
\end{bmatrix}$ and ${\bf b} = \begin{bmatrix} -1 \\ ~~3 \\ -1 \\ ~~4 \end{bmatrix}$. Is ${\bf b}$ in the
range of the linear transformation $T({\bf x}) = A{\bf x} $ ?
\vskip 10pt
\noindent
3. Let $T : {\Bbb R}^n \rightarrow {\Bbb R}^m$ be a linear transformation, and let 
$\left\{ {\bf v}_1, {\bf v}_2, {\bf v_3} \right \}$ be a linearly dependent set in ${\Bbb R}^n$. Explain
why the set $\left\{ T({\bf v}_1), T({\bf v}_2), T({\bf v_3}) \right \}$ is linearly dependent.
\vskip 10pt
\noindent
4. Consider a linear transformation from $T : {\Bbb R}^3 \rightarrow {\Bbb R}^2$, where 
\[
T\begin{bmatrix} 1 \\ 0 \\ 0 \end{bmatrix} = \begin{bmatrix} 7 \\ 11 \end{bmatrix},\hskip 15pt T\begin{bmatrix} 0 \\ 1 \\ 0 \end{bmatrix} = \begin{bmatrix} 6 \\ 9 \end{bmatrix},\hskip 15pt \mbox{and} \hskip 15pt
T\begin{bmatrix} 0 \\ 0 \\ 1 \end{bmatrix} = \begin{bmatrix} -13\\~~17\end{bmatrix}.
\]
Find the standard matrix $A$ of the transformation $T$.
\vskip 10pt
\noindent
5. Let $T: {\Bbb R}^2 \rightarrow {\Bbb R}^3$ be a linear transformation such that \\
$T(x_1,x_2) = (x_1-2x_2,-x_1+3x_2,3x_1-2x_2).$ Find ${\bf x}$ such that $T({\bf x}) = (-1,4,9).$
\vskip 10pt
\noindent
6. Find the standard matrix for the linear transformation $T: {\Bbb R}^2 \rightarrow {\Bbb R}^2$
which is a horizontal shear transformation that leaves ${\bf e}_1$ unchanged and maps
${\bf e}_2$ into ${\bf e}_2 + 3{\bf e}_1$.
\vskip 10pt
\noindent
7. The color of light can be represented in a vector $\begin{bmatrix} R \\ G \\ B \end{bmatrix}$ where $R = \text{amount of red}$,
$G = \text{amount of green}$, and $B = \text{amount of blue}$. The human eye and the brain transform the incoming signal into the
signal $\begin{bmatrix} I \\ L \\ S\end{bmatrix}$, where
\[
\begin{matrix}
\text{~~~~~~~~~~intensity} & I & = & \frac{R+G+B}{3}\\
\text{long-wave signal} & L & = & R - G \\
\text{short-wave signal} & S & =& B - \frac{R+G}{2}.
\end{matrix}
\]
\vskip 10pt
\noindent
(a) Find the matrix $P$ representing the transformation from  $\begin{bmatrix} R \\ G \\ B \end{bmatrix}$ to
$\begin{bmatrix} I \\ L \\ S\end{bmatrix}$
\vskip 5pt
\noindent
(b) Consider a pair of yellow sunglasses for water sports which cuts out all blue light and passes all red and green light. Find the
matrix $A$ which represents the transformation incoming light undergoes as it passes through the sunglasses.
\vskip 5pt
\noindent
(c) Find the matrix for the composite transformation which light undergoes as it first passes through the sunglasses and then the eye.
\vskip 10pt
\noindent
8.~ Let ${\bf v}$ be a fixed vector in ${\mathbb R}^n$ and let $T: {\mathbb R}^n \rightarrow {\mathbb R}$ be the mapping defined
by $T({\bf x}) = {\bf v}^T {\bf x}$ (i.e. the standard inner product).
\begin{itemize}
\item[(a)] Is $T$ a linear operator?
\item[(b)] Is $T$ a linear transformation?
\end{itemize}
\vskip 10pt
\noindent
9.~  Find the $ 3 \times 3$ matrices that produce the described composite 2D transformations, using homogeneous coordinates. Apply the transformations to the {\bf letter N} data, ``letterN.pny" and submit the corresponding plots as well.
\begin{itemize}
\item[(a)] Translate by $(-2, 3)$, and then scale the $x$-coordinate by $0.8$ and the $y$-coordinate by $1.2$
\item[(b)] Rotate points $\frac{\pi}{6}$, and then reflect through the $x$-axis.
\end{itemize} 


\end{document}

